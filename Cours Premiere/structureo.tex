%----------------------------------------------------------------------------------------
%	VARIOUS REQUIRED PACKAGES AND CONFIGURATIONS
%----------------------------------------------------------------------------------------

\usepackage[top=3cm,bottom=3cm,left=3cm,right=3cm,headsep=1cm,a4paper]{geometry} % Page margins

\usepackage{graphicx} % Required for including pictures
\graphicspath{{Pictures/}} % Specifies the directory where pictures are stored

\usepackage{wrapfig}

\usepackage{lipsum} % Inserts dummy text


\usepackage[french]{babel} % French language/hyphenation

\usepackage[utf8]{inputenc}

\usepackage{enumitem} % Customize lists
\setlist{nolistsep} % Reduce spacing between bullet points and numbered lists

\usepackage{booktabs} % Required for nicer horizontal rules in tables

\usepackage{textcomp} % For straight single quotes

\usepackage{caption} % For caption*

\usepackage{xcolor} % Required for specifying colors by name
\definecolor{Hred}{HTML}{B21353} % Define the orange color used for highlighting throughout the book
\definecolor{Hblue}{HTML}{002451}

\usepackage[hyperfootnotes=false]{hyperref} % For hyperlinks
\hypersetup{
	colorlinks,
	linkcolor={Hred},
	urlcolor={Hblue}
}

\usepackage{multicol}
\newcommand{\DANGER}{\noindent{\color{Hred}\danger} }

%----------------------------------------------------------------------------------------
%	TIKZs
%----------------------------------------------------------------------------------------

\usepackage{tikz,pgfplots,tkz-tab,tkz-fct}
\usetikzlibrary{arrows.meta}
\usetikzlibrary{shapes.misc}
\usetikzlibrary{tikzmark}
\usepackage{tkz-euclide}
\usetkzobj{all}

\usetikzlibrary{lindenmayersystems}
\usetikzlibrary{matrix,arrows,decorations.pathmorphing}
\usetikzlibrary{shapes,snakes}
\usetikzlibrary{positioning}
\usetikzlibrary{calc}
\usetikzlibrary{patterns}
\usetikzlibrary{arrows}
\usetikzlibrary{intersections,through,backgrounds}

\usepgfplotslibrary{fillbetween}

\pgfdeclarelayer{bg}
\pgfsetlayers{bg,main}

\tikzset{
	hatch distance/.store in=\hatchdistance,
	hatch distance=20pt,
	hatch thickness/.store in=\hatchthickness,
	hatch thickness=0.3pt
}

\makeatletter
\pgfdeclarepatternformonly[\hatchdistance,\hatchthickness]{northeast}
{\pgfqpoint{0pt}{0pt}}
{\pgfqpoint{\hatchdistance}{\hatchdistance}}
{\pgfpoint{\hatchdistance-1pt}{\hatchdistance-1pt}}%
{
	\pgfsetlinewidth{\hatchthickness}
	\pgfpathmoveto{\pgfqpoint{0pt}{0pt}}
	\pgfpathlineto{\pgfqpoint{\hatchdistance}{\hatchdistance}}
	\pgfusepath{stroke}
}

\pgfdeclarepatternformonly[\hatchdistance,\hatchthickness]{northwest}
{\pgfqpoint{0pt}{0pt}}
{\pgfqpoint{\hatchdistance}{\hatchdistance}}
{\pgfpoint{\hatchdistance-1pt}{\hatchdistance-1pt}}%
{
	\pgfsetlinewidth{\hatchthickness}
	\pgfpathmoveto{\pgfqpoint{\hatchdistance}{0pt}}
	\pgfpathlineto{\pgfqpoint{0pt}{\hatchdistance}}
	\pgfusepath{stroke}
}

\pgfdeclarepatternformonly[\hatchdistance,\hatchthickness]{horizontal}
{\pgfqpoint{0pt}{0pt}}
{\pgfqpoint{\hatchdistance}{\hatchdistance}}
{\pgfpoint{\hatchdistance-1pt}{\hatchdistance-1pt}}%
{
	\pgfsetcolor{\tikz@pattern@color}
	\pgfsetlinewidth{\hatchthickness}
	\pgfpathmoveto{\pgfqpoint{0pt}{0pt}}
	\pgfpathlineto{\pgfqpoint{\hatchdistance}{0pt}}
	\pgfusepath{stroke}
}

\pgfdeclarepatternformonly[\hatchdistance,\hatchthickness]{vertical}
{\pgfqpoint{0pt}{0pt}}
{\pgfqpoint{\hatchdistance}{\hatchdistance}}
{\pgfpoint{\hatchdistance-1pt}{\hatchdistance-1pt}}%
{
	\pgfsetcolor{\tikz@pattern@color}
	\pgfsetlinewidth{\hatchthickness}
	\pgfpathmoveto{\pgfqpoint{0pt}{0pt}}
	\pgfpathlineto{\pgfqpoint{0pt}{\hatchdistance}}
	\pgfusepath{stroke}
}
\makeatother


%----------------------------------------------------------------------------------------
%	Tabular
%----------------------------------------------------------------------------------------


\usepackage{multirow}
\usepackage{array}
\newcolumntype{L}[1]{>{\raggedright\let\newline\\\arraybackslash\hspace{0pt}}m{#1}}
\newcolumntype{C}[1]{>{\centering\let\newline\\\arraybackslash\hspace{0pt}}m{#1}}
\newcolumntype{R}[1]{>{\raggedleft\let\newline\\\arraybackslash\hspace{0pt}}m{#1}}
\newcolumntype{N}{@{}m{0pt}@{}}

\usepackage{diagbox} % For 2 way tables

\newcommand{\bi}[1]{\textbf{\textit{#1}}}


%----------------------------------------------------------------------------------------
%	FONTS & TEXT
%----------------------------------------------------------------------------------------

\usepackage{avant} % Use the Avantgarde font for headings
%\usepackage{times} % Use the Times font for headings
\usepackage{mathptmx} % Use the Adobe Times Roman as the default text font together with math symbols from the Sym­bol, Chancery and Com­puter Modern fonts

\usepackage{microtype} % Slightly tweak font spacing for aesthetics
\usepackage[utf8]{inputenc} % Required for including letters with accents
\usepackage[T1]{fontenc} % Use 8-bit encoding that has 256 glyphs

\usepackage{setspace} % Change line spacing for main text in document
\setstretch{1.25}

%----------------------------------------------------------------------------------------
%	MAIN TABLE OF CONTENTS
%----------------------------------------------------------------------------------------

\usepackage{titletoc} % Required for manipulating the table of contents

\contentsmargin{0cm} % Removes the default margin

%----------------------------------------------------------------------------------------
%	THEOREM STYLES
%----------------------------------------------------------------------------------------

\usepackage{amsmath,amsfonts,amssymb,amsthm} % For math equations, theorems, symbols, etc


% Boxed/framed environments
\newtheoremstyle{Hrednumbox}% % Theorem style name
{0pt}% Space above
{0pt}% Space below
{\normalfont}% % Body font
{}% Indent amount
{\small\bf\sffamily\color{Hred}}% % Theorem head font
{}% Punctuation after theorem head
{0.25em}% Space after theorem head
{\small\sffamily\color{Hred}~{#1}~{#2}.}% Theorem text (e.g. Theorem 2.1)

% Non-boxed/non-framed environments
\newtheoremstyle{Hrednum}% % Theorem style name
{5pt}% Space above
{5pt}% Space below
{\normalfont}% % Body font
{}% Indent amount
{\small\bf\sffamily\color{Hred}}% % Theorem head font
{}% Punctuation after theorem head
{0.25em}% Space after theorem head
{\small\sffamily\color{Hred}~{#1}~{#2}.}% Theorem text (e.g. Theorem 2.1)

% Blue, no space
\newtheoremstyle{Hbluenum}% % Theorem style name
{0pt}% Space above
{0pt}% Space below
{\normalfont}% % Body font
{}% Indent amount
{\small\bf\sffamily\color{Hblue}}% % Theorem head font
{}% Punctuation after theorem head
{0.25em}% Space after theorem head
{\small\sffamily\color{Hblue}~{#1}~{#2}.}% Theorem text (e.g. Theorem 2.1)

\renewcommand{\qedsymbol}{$\blacksquare$}% Optional qed square
\makeatother

\newtheoremstyle{blacknumex}% Theorem style name
{5pt}% Space above
{5pt}% Space below
{\normalfont}% Body font
{} % Indent amount
{\small\bf\sffamily}% Theorem head font
{}% Punctuation after theorem head
{0.25em}% Space after theorem head
{\small\sffamily{\tiny\ensuremath{\blacksquare}}~{#1}~{#2}.}% Theorem text (e.g. Theorem 2.1)}

\newtheoremstyle{blacknumbox} % Theorem style name
{0pt}% Space above
{0pt}% Space below
{\normalfont}% Body font
{}% Indent amount
{\small\bf\sffamily}% Theorem head font
{}% Punctuation after theorem head
{0.25em}% Space after theorem head
{\small\sffamily~{#1}~{#2}.}% Theorem text (e.g. Theorem 2.1)


% Defines the theorem text style for each type of theorem to one of the three styles above
%\newcounter{dummy} 
%\numberwithin{dummy}{chapter}

\theoremstyle{Hrednumbox}
\newtheorem{theoremeT}{Théorème}[chapter]
\newtheorem{problem}{Problem}[chapter]
\newtheorem{definitionT}{Définition}[chapter]

\theoremstyle{blacknumex}
\newtheorem{exampleT}{Exemple}[chapter]
\newtheorem{exerciseT}{Exercice}[chapter]
\newtheorem{correctionT}{Correction}[chapter]

\theoremstyle{blacknumbox}
\newtheorem{vocabulary}{Vocabulaire}[chapter]
\newtheorem{corollaryT}{Corollaire}[chapter]


\theoremstyle{Hbluenum}
\newtheorem{propositionT}{Propriété}[chapter]

%----------------------------------------------------------------------------------------
%	DEFINITION OF COLORED BOXES
%----------------------------------------------------------------------------------------

\RequirePackage[framemethod=default]{mdframed} % Required for creating the theorem, definition, exercise and corollary boxes

% Theorem box
\newmdenv[skipabove=7pt,
skipbelow=7pt,
backgroundcolor=black!5,
linecolor=Hred,
innerleftmargin=5pt,
innerrightmargin=5pt,
innertopmargin=5pt,
leftmargin=0cm,
rightmargin=0cm,
innerbottommargin=5pt]{tBox}

% Exercise box	  
\newmdenv[skipabove=7pt,
skipbelow=7pt,
rightline=false,
leftline=true,
topline=false,
bottomline=false,
backgroundcolor=Hred!5,
linecolor=Hred,
innerleftmargin=5pt,
innerrightmargin=5pt,
innertopmargin=5pt,
innerbottommargin=5pt,
leftmargin=0cm,
rightmargin=0cm,
linewidth=4pt]{eBox}	

% Definition box
\newmdenv[skipabove=7pt,
skipbelow=7pt,
rightline=false,
leftline=true,
topline=false,
bottomline=false,
linecolor=Hred,
innerleftmargin=5pt,
innerrightmargin=5pt,
innertopmargin=0pt,
leftmargin=0cm,
rightmargin=0cm,
linewidth=4pt,
innerbottommargin=0pt]{dBox}	

% Corollary box
\newmdenv[skipabove=7pt,
skipbelow=7pt,
rightline=false,
leftline=true,
topline=false,
bottomline=false,
linecolor=gray,
backgroundcolor=black!5,
innerleftmargin=5pt,
innerrightmargin=5pt,
innertopmargin=5pt,
leftmargin=0cm,
rightmargin=0cm,
linewidth=4pt,
innerbottommargin=5pt]{cBox}

% Proposition box
\newmdenv[skipabove=7pt,
skipbelow=7pt,
rightline=false,
leftline=true,
topline=false,
bottomline=false,
linecolor=Hblue,
backgroundcolor=Hblue!5,
innerleftmargin=5pt,
innerrightmargin=5pt,
innertopmargin=5pt,
leftmargin=0cm,
rightmargin=0cm,
linewidth=4pt,
innerbottommargin=5pt]{pBox}

% Creates an environment for each type of theorem and assigns it a theorem text style from the "Theorem Styles" section above and a colored box from above
\newenvironment{theorem}{\begin{tBox}\begin{theoremeT}}{\end{theoremeT}\end{tBox}}
\newenvironment{exercise}{\begin{eBox}\begin{exerciseT}}{\hfill{\color{Hred}\tiny\ensuremath{\blacksquare}}\end{exerciseT}\end{eBox}}
\newenvironment{definition}{\begin{eBox}\begin{definitionT}}{\end{definitionT}\end{eBox}}	
\newenvironment{example}{\begin{exampleT}}{\hfill{\tiny\ensuremath{\blacksquare}}\end{exampleT}}	
\newenvironment{correction}{\begin{correctionT}}{\hfill{\tiny\ensuremath{\blacksquare}}\end{correctionT}}	
\newenvironment{corollary}{\begin{cBox}\begin{corollaryT}}{\end{corollaryT}\end{cBox}}	
\newenvironment{proposition}{\begin{pBox}\begin{propositionT}}{\end{propositionT}\end{pBox}}
%----------------------------------------------------------------------------------------
%	REMARK ENVIRONMENT
%----------------------------------------------------------------------------------------

\newenvironment{remark}{\par\vspace{10pt}\small % Vertical white space above the remark and smaller font size
\begin{list}{}{
\leftmargin=35pt % Indentation on the left
\rightmargin=25pt}\item\ignorespaces % Indentation on the right
\makebox[-2.5pt]{\begin{tikzpicture}[overlay]
\node[draw=Hred!60,line width=1pt,circle,fill=Hred!25,font=\sffamily\bfseries,inner sep=2pt,outer sep=0pt] at (-15pt,0pt){\textcolor{Hred}{R}};\end{tikzpicture}} % Orange R in a circle
\advance\baselineskip -1pt}{\end{list}\vskip5pt} % Tighter line spacing and white space after remark

%----------------------------------------------------------------------------------------
%	SECTION NUMBERING IN THE MARGIN
%----------------------------------------------------------------------------------------

\makeatletter
\renewcommand{\@seccntformat}[1]{\llap{\textcolor{Hred}{\csname the#1\endcsname}\hspace{1em}}}                    
\renewcommand{\section}{\@startsection{section}{1}{\z@}
{-4ex \@plus -1ex \@minus -.4ex}
{1ex \@plus.2ex }
{\normalfont\large\sffamily\bfseries}}
\renewcommand{\subsection}{\@startsection {subsection}{2}{\z@}
{-3ex \@plus -0.1ex \@minus -.4ex}
{0.5ex \@plus.2ex }
{\normalfont\sffamily\bfseries}}
\renewcommand{\subsubsection}{\@startsection {subsubsection}{3}{\z@}
{-2ex \@plus -0.1ex \@minus -.2ex}
{.2ex \@plus.2ex }
{\normalfont\small\sffamily\bfseries}}                        
\renewcommand\paragraph{\@startsection{paragraph}{4}{\z@}
{-2ex \@plus-.2ex \@minus .2ex}
{.1ex}
{\normalfont\small\sffamily\bfseries}}

\renewcommand\thesection{\Roman{section}.} %redefine numbering before section title
\renewcommand\thesubsection{\arabic{subsection}.}

%----------------------------------------------------------------------------------------
%	CHAPTER HEADINGS
%----------------------------------------------------------------------------------------



\addto\captionsfrench{\renewcommand{\chaptername}{Semaine}}

\usepackage{titlesec}

\titlespacing*{\chapter}{0pt}{-25pt}{40pt}
\titleformat{\chapter}[block]{\color{Hblue}\LARGE\bfseries\filcenter}{\color{Hred}\Huge\it\bfseries Semaine \thechapter}{0pt}{\\}


%----------------------------------------------------------------------------------------
%	MATH STUFF
%----------------------------------------------------------------------------------------


\usepackage{array}
\newcolumntype{L}[1]{>{\raggedright\let\newline\\\arraybackslash\hspace{0pt}}m{#1}}
\newcolumntype{C}[1]{>{\centering\let\newline\\\arraybackslash\hspace{0pt}}m{#1}}
\newcolumntype{R}[1]{>{\raggedleft\let\newline\\\arraybackslash\hspace{0pt}}m{#1}}

\usepackage{pbox}

\usepackage{bbm}

\usepackage{xcolor,colortbl}

\usepackage{pgfplots}

\usepackage{mathtools}

\usepackage{stmaryrd} %double square brackets

\usepackage{cancel}

\usepackage[utf8]{inputenc}
\usepackage[upright]{fourier}
\usepackage{tikz}
\usetikzlibrary{matrix,arrows,decorations.pathmorphing}
\usetikzlibrary{shapes,snakes}
\usetikzlibrary{positioning}
\usetikzlibrary{calc}
\usetikzlibrary{patterns}
\usetikzlibrary{arrows}


\usepackage{graphicx}
\usepackage{wrapfig}


\newcommand{\fonction}[5]{\begin{array}{lrcl}
		#1: & #2 & \longrightarrow & #3 \\
		& #4 & \longmapsto & #5 \end{array}}

\newcommand{\fonctionsans}[4]{\begin{array}{rcl}
		#1 & \longrightarrow & #2 \\
		#3 & \longmapsto & #4 \end{array}}

\newcommand{\intent}[2]{\llbracket #1, #2 \rrbracket}

\newcommand{\Lim}[1]{\raisebox{0.5ex}{\scalebox{0.8}{$\displaystyle \lim_{#1}\;$}}}
\newcommand{\limninf}{\Lim{n\rightarrow +\infty}}
\newcommand{\limx}[2]{\Lim{#1\rightarrow #2}}


\newcommand{\R}{\mathbb{R}}
\newcommand{\N}{\mathbb{N}}
\newcommand{\Z}{\mathbb{Z}}
\newcommand{\C}{\mathbb{C}}
\newcommand{\Q}{\mathbb{Q}}
\newcommand{\K}{\mathbb{K}}
\newcommand{\KX}{\K[X]}
\newcommand{\KnX}{\K_n[X]}
\newcommand{\Prob}[1]{\mathbb{P}\left(#1\right)}
\newcommand{\card}[1]{\textnormal{card}\left(#1\right)}

\newcommand{\F}{\mathcal{F}}
\newcommand{\vect}[1]{\textsc{Vect}\left( #1 \right)}

\newcommand{\B}{\mathcal{B}}
\newcommand{\Btn}{\mathcal{B}=\left(b_1,...,b_n\right)}

\newcommand{\Df}{\mathcal{D}_f}


\newcommand{\Rbar}{\overline{\R}}

\newcommand{\V}{\forall}
\newcommand{\E}{\exists}

\newcommand{\inR}{\in\R}
\newcommand{\inN}{\in\N}
\newcommand{\inZ}{\in\Z}
\newcommand{\inC}{\in\C}
\newcommand{\inQ}{\in\Q}

\newcommand{\vu}{\overrightarrow{u}}
\newcommand{\vv}{\overrightarrow{v}}
\newcommand{\vpts}[2]{\overrightarrow{#1#2}}
\newcommand{\vn}{\overrightarrow{0}}
\newcommand{\vdet}[2]{\det\left(\overrightarrow{#1},\overrightarrow{#2}\right)}

\newcommand{\VninN}{\V n \inN}
\newcommand{\VxinR}{\V x \inR}
\newcommand{\Veps}{\V \varepsilon> 0}


\newcommand{\floor}[1]{\lfloor #1 \rfloor}
\newcommand{\ceil}[1]{\lceil #1 \rceil}

\newcommand{\MnK}{\mathcal{M}_n(\K)}

\newcommand{\diag}{\textnormal{Diag}}

\def\G{{\frak S}}

\newcommand{\EPbF}{\left(\Omega,\mathcal{P}(\Omega)\right)}

\newcommand{\EPF}{\left(\Omega,\mathcal{P}(\Omega),P\right)}

\newcommand{\EP}{\left(\Omega,\mathcal{A},P\right)}

\newcommand{\egloi}{\stackrel{\mathcal{L}}{=}}

\newcommand{\rg}{\textnormal{rg}}

\newcommand{\im}{\textnormal{im}}

% l' unite
\newcommand{\myunit}{1 cm}
\tikzset{
	node style sp/.style={draw,circle,minimum size=\myunit},
	node style ge/.style={circle,minimum size=\myunit},
	arrow style mul/.style={draw,sloped,midway,fill=white},
	arrow style plus/.style={midway,sloped,fill=white},
}


%----------------------------------------------------------------------------------------
%	PYTHON
%----------------------------------------------------------------------------------------


\usepackage{xcolor}
\definecolor{maroon}{cmyk}{0, 0.87, 0.68, 0.32}
\definecolor{halfgray}{gray}{0.55}
\definecolor{ipython_frame}{RGB}{207, 207, 207}
\definecolor{ipython_bg}{RGB}{247, 247, 247}
\definecolor{ipython_red}{RGB}{186, 33, 33}
\definecolor{ipython_green}{RGB}{0, 128, 0}
\definecolor{ipython_cyan}{RGB}{64, 128, 128}
\definecolor{ipython_purple}{RGB}{170, 34, 255}

\usepackage{listingsutf8}

\lstset{inputencoding=utf8/latin1}

\lstset{
	breaklines=true,
	%
	extendedchars=true,
	literate=
	{á}{{\'a}}1 {é}{{\'e}}1 {í}{{\'i}}1 {ó}{{\'o}}1 {ú}{{\'u}}1
	{Á}{{\'A}}1 {É}{{\'E}}1 {Í}{{\'I}}1 {Ó}{{\'O}}1 {Ú}{{\'U}}1
	{à}{{\`a}}1 {è}{{\`e}}1 {ì}{{\`i}}1 {ò}{{\`o}}1 {ù}{{\`u}}1
	{À}{{\`A}}1 {È}{{\'E}}1 {Ì}{{\`I}}1 {Ò}{{\`O}}1 {Ù}{{\`U}}1
	{ä}{{\"a}}1 {ë}{{\"e}}1 {ï}{{\"i}}1 {ö}{{\"o}}1 {ü}{{\"u}}1
	{Ä}{{\"A}}1 {Ë}{{\"E}}1 {Ï}{{\"I}}1 {Ö}{{\"O}}1 {Ü}{{\"U}}1
	{â}{{\^a}}1 {ê}{{\^e}}1 {î}{{\^i}}1 {ô}{{\^o}}1 {û}{{\^u}}1
	{Â}{{\^A}}1 {Ê}{{\^E}}1 {Î}{{\^I}}1 {Ô}{{\^O}}1 {Û}{{\^U}}1
	{œ}{{\oe}}1 {Œ}{{\OE}}1 {æ}{{\ae}}1 {Æ}{{\AE}}1 {ß}{{\ss}}1
	{ç}{{\c c}}1 {Ç}{{\c C}}1 {ø}{{\o}}1 {å}{{\r a}}1 {Å}{{\r A}}1
	{€}{{\EUR}}1 {£}{{\pounds}}1
}

\lstdefinelanguage{iPython}{
	morekeywords={access,and,break,class,continue,def,del,elif,else,except,exec,finally,for,from,global,if,import,in,is,lambda,not,or,pass,print,raise,return,try,while},%
	%
	% Built-ins
	morekeywords=[2]{abs,all,any,basestring,bin,bool,bytearray,callable,chr,classmethod,cmp,compile,complex,delattr,dict,dir,divmod,enumerate,eval,execfile,file,filter,float,format,frozenset,getattr,globals,hasattr,hash,help,hex,id,input,int,isinstance,issubclass,iter,len,list,locals,long,map,max,memoryview,min,next,object,oct,open,ord,pow,property,range,raw_input,reduce,reload,repr,reversed,round,set,setattr,slice,sorted,staticmethod,str,sum,super,tuple,type,unichr,unicode,vars,xrange,zip,apply,buffer,coerce,intern},%
	%
	sensitive=true,%
	morecomment=[l]\#,%
	morestring=[b]',%
	morestring=[b]",%
	%
	morestring=[s]{'''}{'''},% used for documentation text (mulitiline strings)
	morestring=[s]{"""}{"""},% added by Philipp Matthias Hahn
	%
	morestring=[s]{r'}{'},% `raw' strings
	morestring=[s]{r"}{"},%
	morestring=[s]{r'''}{'''},%
	morestring=[s]{r"""}{"""},%
	morestring=[s]{u'}{'},% unicode strings
	morestring=[s]{u"}{"},%
	morestring=[s]{u'''}{'''},%
	morestring=[s]{u"""}{"""},%
	%
	% {replace}{replacement}{lenght of replace}
	% *{-}{-}{1} will not replace in comments and so on
	literate=
	{á}{{\'a}}1 {é}{{\'e}}1 {í}{{\'i}}1 {ó}{{\'o}}1 {ú}{{\'u}}1
	{Á}{{\'A}}1 {É}{{\'E}}1 {Í}{{\'I}}1 {Ó}{{\'O}}1 {Ú}{{\'U}}1
	{à}{{\`a}}1 {è}{{\`e}}1 {ì}{{\`i}}1 {ò}{{\`o}}1 {ù}{{\`u}}1
	{À}{{\`A}}1 {È}{{\'E}}1 {Ì}{{\`I}}1 {Ò}{{\`O}}1 {Ù}{{\`U}}1
	{ä}{{\"a}}1 {ë}{{\"e}}1 {ï}{{\"i}}1 {ö}{{\"o}}1 {ü}{{\"u}}1
	{Ä}{{\"A}}1 {Ë}{{\"E}}1 {Ï}{{\"I}}1 {Ö}{{\"O}}1 {Ü}{{\"U}}1
	{â}{{\^a}}1 {ê}{{\^e}}1 {î}{{\^i}}1 {ô}{{\^o}}1 {û}{{\^u}}1
	{Â}{{\^A}}1 {Ê}{{\^E}}1 {Î}{{\^I}}1 {Ô}{{\^O}}1 {Û}{{\^U}}1
	{œ}{{\oe}}1 {Œ}{{\OE}}1 {æ}{{\ae}}1 {Æ}{{\AE}}1 {ß}{{\ss}}1
	{ç}{{\c c}}1 {Ç}{{\c C}}1 {ø}{{\o}}1 {å}{{\r a}}1 {Å}{{\r A}}1
	{€}{{\EUR}}1 {£}{{\pounds}}1
	%
	{^}{{{\color{ipython_purple}\^{}}}}1
	{=}{{{\color{ipython_purple}=}}}1
	%
	{+}{{{\color{ipython_purple}+}}}1
	{*}{{{\color{ipython_purple}$^\ast$}}}1
	{/}{{{\color{ipython_purple}/}}}1
	%
	{+=}{{{+=}}}1
	{-=}{{{-=}}}1
	{*=}{{{$^\ast$=}}}1
	{/=}{{{/=}}}1,
	literate=
	*{-}{{{\color{ipython_purple}-}}}1
	{?}{{{\color{ipython_purple}?}}}1,
	%
	identifierstyle=\color{black}\ttfamily,
	commentstyle=\color{ipython_cyan}\ttfamily,
	stringstyle=\color{ipython_red}\ttfamily,
	keepspaces=true,
	showspaces=false,
	showstringspaces=false,
	%
	rulecolor=\color{ipython_frame},
	frame=single,
	frameround={t}{t}{t}{t},
	numbers=left,
	numberstyle=\footnotesize\color{halfgray},
	numbersep=1em,
	xleftmargin=1em,
	framextopmargin=0em,
	framexbottommargin=0em,
	%
	%
	backgroundcolor=\color{ipython_bg},
	%   extendedchars=true,
	basicstyle=\ttfamily\small\setstretch{1},
	keywordstyle=\color{ipython_green}\ttfamily,
	inputencoding=utf8/latin1,
}

\lstdefinelanguage{Pseudo}{
	numbers=left,
	numberstyle=\footnotesize,
	numbersep=1em,
	xleftmargin=1em,
	framextopmargin=0em,
	framexbottommargin=0em,
	showspaces=false,
	showtabs=false,
	showstringspaces=false,
	frame=lrtb,
	tabsize=4,
	inputencoding=utf8/latin1,
	% Basic
	basicstyle=\ttfamily\small\setstretch{1},
	backgroundcolor=\color{white},
	% Comments
	commentstyle=\color{Comments}\slshape,
	% Strings
	stringstyle=\color{red},
	morecomment=[s][\color{Strings}]{"""}{"""},
	morecomment=[s][\color{Strings}]{'''}{'''},
	% keywords
	% additional keywords
	literate=
	{=}{$\leftarrow{}$}{1}
	{==}{$={}$}{1}
	{á}{{\'a}}1 {é}{{\'e}}1 {í}{{\'i}}1 {ó}{{\'o}}1 {ú}{{\'u}}1
	{Á}{{\'A}}1 {É}{{\'E}}1 {Í}{{\'I}}1 {Ó}{{\'O}}1 {Ú}{{\'U}}1
	{à}{{\`a}}1 {è}{{\`e}}1 {ì}{{\`i}}1 {ò}{{\`o}}1 {ù}{{\`u}}1
	{À}{{\`A}}1 {È}{{\'E}}1 {Ì}{{\`I}}1 {Ò}{{\`O}}1 {Ù}{{\`U}}1
	{ä}{{\"a}}1 {ë}{{\"e}}1 {ï}{{\"i}}1 {ö}{{\"o}}1 {ü}{{\"u}}1
	{Ä}{{\"A}}1 {Ë}{{\"E}}1 {Ï}{{\"I}}1 {Ö}{{\"O}}1 {Ü}{{\"U}}1
	{â}{{\^a}}1 {ê}{{\^e}}1 {î}{{\^i}}1 {ô}{{\^o}}1 {û}{{\^u}}1
	{Â}{{\^A}}1 {Ê}{{\^E}}1 {Î}{{\^I}}1 {Ô}{{\^O}}1 {Û}{{\^U}}1
	{œ}{{\oe}}1 {Œ}{{\OE}}1 {æ}{{\ae}}1 {Æ}{{\AE}}1 {ß}{{\ss}}1
	{ç}{{\c c}}1 {Ç}{{\c C}}1 {ø}{{\o}}1 {å}{{\r a}}1 {Å}{{\r A}}1
	{€}{{\EUR}}1 {£}{{\pounds}}1,
	morekeywords={Pour, Fin, Faire,Si, Alors, Sinon, est, vraie, si, pour, faire, Afficher, allant, de, Tant, tant, que},
	keywordstyle=\ttfamily\bf,
	emph={self},
	emphstyle={\color{self}\slshape},
	%
	breaklines=true
}

\usepackage{adjustbox}
\usepackage{varwidth}